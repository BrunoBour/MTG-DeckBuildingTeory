Cartas que normalmente são usadas em contextos errados



\subsection{\card{Chromatic Lantern}:} A lanterna cromática é a mais comum das pedras de mana de cmc3 utilizadas, essa categoria de pedras de mana tende a ser muito mais fraca do que as genéricas de cmc2, e o motivo disso é na verdade bem óbvio, possuem um custo de mana 50\% maior em relação a concorrência, além de ocuparem o cast do terceiro turno, muitas vezes o turno mais importante do Early Game devido a cartas capazes de pressionar o board com consistencia começarem a surgir, dessa forma, toda mana rock cmc3 precisa de uma ótima compensação para ter seu slot justificado.
Assim, ela parece chamar muita atenção devido a capacidade de corrigir cores, porém em muitos casos isso se trata de uma armadilha, uma vez que pedras de cmc2 normalmente já ajudam nessa correção, gerando duas cores distintas ou até mesmo mana de qualquer cor sem possuir o drawback do custo, isto somado a uma base de mana bem construída deveria já ser o suficiente para atingir os pips necessários. 
Se você ainda acredita que precisa da lanterna para corrigir suas cores provavelmente significa que algo está errado, pois se ela é assim necessária, quando você não compra-la irá fazer o que?! não ter mana para castar as spells?! se esse é o caso, acredito que o seu deck possui entre 4-5 cores e tem grande dependêcia dos pips, situação que a torna justificável, mas ainda não o suficiente, tendo em mente que ela não irá corrigir seus pips em toda partida. 
Em conclusão a lanterna precisa estar acompanhada de outras cartas capazes de corrigir sua mana base no único caso em que é viável, tornando-a uma peça importante na situação supracitada, mas uma má eficiência de mana em qualquer outro contexto.

\subsection{\card{Propaganda}:} Propaganda assim como \card{Ghostly Prision}, possui dois usos principais, evitar swings no early game onde o custo da taxa é muito alto e o proteger de swarm decks.
O segundo uso pode lhe salvar em alguns casos, mas não é a ferramenta ideal para o serviço, pois deixa a desejar na confiabilidade devido ao adversário poder simplesmente remove-lo e então realizar o ataque.
Agora, falando do primeiro caso, cumpre muito bem a promessa, stallando creature based decks e poupando uma imensa quantidade de vida, conquista que pode salvar jogos diante de decks que tendem a oprimir seus life points.
Perfeito, mas então por que se trata de uma armadilha?! Bom, justamente pelos jogadores experienciarem essas boas performances da carta cogitam utiliza-la em decks que não precisam dessa proteção.
O exemplo mais corriqueiro são mid-ranges/aggros que jogam na curva, eles já possuem criaturas poderosas capazes de bloquear, fazendo com que se receba muito menos ataques ou até nenhum.
Nesse contexo a propaganda é uma carta que não possui uso algum, e pode ser muito custosa para o early game, seja pelo fato de ocupar o terceiro turno inteiro ou apenas por ocupar um espaço na mão que poderia ser de uma criatura opressora.
De tal maneira apenas dedique um slot a essa carta se o deck realmente precisa de proteção no early game e tem dificuldades para se manter saudável até o seu spike.

\subsection{\card{Maskwood Nexus}:} Se trata de uma carta muito popular que tende a ser utilizada em um contexto equivocado, pode parecer esquisito a primeira mão, mas o seu espaço não é em tribais com bastante suporte.
O motivo disso é que tribais com bastante suporte normalmente já possuem um número consistente de criaturas fortes dentro da tribo, e por exemplo, não faz sentido algum transformar suas criaturas em elfos se 80\%+ delas já são, ainda mais por um alto custo de mana.
Seu spot para um bom desempenho é muito esepecífico, como por exemplo decks que se importam com mais de uma tribo, situação em que ela pode gerar muito valor. Nos demais casos pode performar muito abaixo do esperado.