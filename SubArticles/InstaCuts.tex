Cartas que você deve considerar remover do Deck ex: winmore, fakeCombos, perfectSpots

\subsection{Winmore Cards:}
Possivelmente os Slots mais dificeis de identificar, e por consequencia os mais comuns de se ver em listas iniciais de decks sendo construidos. 
Elas dizem respeito as famosas cartas que só lhe ajudam quando você está muito a frente da mesa, mas não o suficiente para fechar o jogo, o exemplo mais concreto é a carta \card{Unnatural Growth}, vamos trabalhar em cima dela.
Dobrar o poder e resistencia de suas criaturas em todo combate parece muito poderoso, imediatamente vem a cabeça um campo com muitas criaturas imensas e poderosas tendo seu poder dobrado se tornando verdadeiros colossos, mas aí que mora o problema.
Essa é a situação ideal, pois em um campo vazio, ou com uma ou duas criaturas, é apenas uma carta de cmc5 que não lhe traz valor algum, dessa forma é necessário estar muito bem no jogo para que ela possa fazer qualquer coisa.
Nesse caso, a carta ainda não se paga, pois no lugar dela, seria muito melhor um Finisher como \card{Overwhelming Stampede}, capaz de terminar o jogo no mesmo turno em que resolve.
E então uma argumentação pode surgir: "Mas o \card{Unnatural Growth} também pode finalizar o jogo", isso é um fato, ele pode, porém faz a tarefa de forma pífia, uma vez que não concede atropelar e além de ser parada por counterspells como o \card{Overwhelming Stampede}, também pode ser removida por inumeras outras cartas, se tornarndo muiiito mais fácil de parar.
Em resumo, a carta é horrível quando se está atrás, e quando se está na frente \st{parece que está atrás} se torna preferível um bom Finisher, sua contribuição é tornar um campo já com criaturas grandes em criaturas com um corpo ainda maior, mas isso não é a solução para ganhar o jogo.
Esse é o conceito de "Win More", cartas que só são úteis quando se está na frente do jogo e ainda por cima não finalizam ele, tendo assim pouco ou nenhum uso. Elas são cartas fortemente aconselhadas de se remover do baralho.

\subsection{Combos Armadilhas:}
Todo jogador adepto a combos já considerou ou colocou algum combo no deck que não encaixava de forma alguma com o baralho, e isso frequentemente ocorre com a carta \card{Laboratory Maniac} por exemplo, uma carta que não possui absolutamente nenhum uso fora de combos.
Essa opção não necessáriamente é ruim, vemos frequentemente em torneios competitivos o uso de \card{Thassa's Oracle}, que se encaixa nessa mesma descrição, porém existe uma importante distinção entre os contextos, a thoracle é utilizada em baralhos com muitos tutores capazes de realizar o combo com uma consistência absurda.
Tarefa dificil de replicar em baralhos casuais de EDH, o que torna a compra de apenas uma das peças do combo muitas vezes indesejada, já que a carta se torna apenas um draw morto que pode influenciar negativamente suas partidas mais do que você espera.
Por isso sujere-se remover cartas que são uteis única e exclusivante dentro de combos, quando os mesmos não podem ser reproduzidos com consistencia.
Caso seja do seu prazer incluir combos no baralho, opte por utilizar cartas que serão úteis mesmo isoldadas do combo, ou então adicione formas de realizar o combo com consistencia para uma abordagem mais competitiva.


\subsection{Cartas inflexiveis:}
Aquelas cartas que ficam com frequência quatro ou cinco turnos na mão sem poderem ser jogadas, seja por você sempre ter uma jogada melhor ou porque ela realmente não pode ser usada.
Esse tipo de carta é um dos mais prejudiciais para a fluidez de um baralho, pois reduz seu número de opções ao ter uma carta a menos na mão e também se trata de um recurso perdido. Uma carta que podia estar mudando o rumo da partida tem seu lugar ocupado por uma carta que sequer pode ser conjurada.
Identificar esses slots e remove-los do baralho é um grande avanço na construção de um bom deck, por isso, ao avaliar uma carta, tenha certeza de que ela é flexivel a ponto de fazer algo em diferentes contextos, e não apenas em uma situação muito específica e pouco recorrente do jogo.
Para identificar essas cartas com mais facilidade preste bastante atenção nos requisitos para conjurar uma carta, por exemplo, ao utilizar um \card{Supreme Verdict} você precisa que as criaturas de seus adversários sejam mais assustadoras que as quais você controla, além de garantir que perder suas criaturas não o deixará fora do jogo.
Outro exemplo é \card{Victimize}, uma carta que exige ter duas boas criaturas no cemitério e ainda por cima uma criatura "fraca" no campo, que valha a pena sacrificar, caso seu deck não consiga dar Fill no cemitério com certa facilidade ou até mesmo conseguir uma boa opção de sacrificio, essa ótima carta, pode se tornar uma carta horrível sem valor algum.
Vale ressaltar também que existem cartas uitilizaveis apenas em situações especificas que podem ser viáveis de se utilizar, basta que a situação não seja tão rara e seu efeito seja extremamente recompensador, então avalie essas vertentes com cuidado.

\subsection{Cartas que desviam o plano de jogo:}
Com certeza todo jogador evita utilizar cartas que não sinergizam com o deck, ninguém colocaria um \card{Ezuri, Renegade Leader} em um deck de Goblins, mas mais do que isso é necessário observar seu plano de jogo e o Tempo em que seu deck funciona para evitar cartas que desviam dessas ideias.
Utilizarei aqui de exemplo o \card{Simian Spirit Guide}, uma carta muito forte e popular que permite sacrificar valor para acelerar seu Tempo, mas ela perde uma contribuição quando utilizada em um deck que precisa de muito recurso para jogar.
Isso acontece pois um deck que depende de Valor para rodar está trocando esse tão prestigiado recurso por Tempo, uma vantagem que não será explorada com eficiência pelo resto do baralho. Por isso é necessário verificar se suas cartas estão dentro da premissa do deck, analizar se realmente fazem o que o deck precisa que façam.
E caso elas estejam deslocadas do seu plano de jogo, faz muito sentido que remova-as para um ter uma ideia de jogo mais estabelecida.


\subsection{Cartas que auxiliam oponentes:}
Esses Slots são muitas vezes fáceis de identificar, mas podem ser muito dificeis de ter o impacto mensurado, isso porque costumam ser cartas muito úteis para seu baralho e lhe ajudar bastante.
O que acontece é que podem estar ajudando seus adversários muito mais do que você pensa. Principalmente quando se está iniciando em um TCG pode não se ter a dimensão do quanto dar um Draw para um oponente muda o jogo.
Existem muitas cartas com efeitos de Howling Mine, como a própria que origina esse nome \card{Howling Mine}, e essas peças podem ser o motivo de sua derrota. Embora pareça promissor ganhar mais recurso e ativar efeitos como o de \card{Xyris, the Writhing Storm} ou \card{Nekusar, the Mindrazer}, tem muitas coisas que é preciso considerar.
Primeiramente é importante perceber que você está gastanto do seu recurso e tempo para colocar essas cartas na mesa, enquanto os oponentes se beneficiam do efeito sem nenhum Drawback, e por segundo, mas ainda mais importante, considerando que você está contra três outros jogadores, note que enquanto você compra uma carta, seus oponentes compram juntos três delas.
Dessa forma se torna muito difícil vencer spots de Archenemy, dando muita margem para que jogadores mais fracos na mesa possam tirar uma vitória "do nada". Permitir essa imprevisibilidade para seus oponentes é um grande erro quando o seu objetivo é ganhar.
Os efeitos de Howling Mine são os mais fáceis de verificar esse problema, mas ainda é comum de acontecer em cartas como \card{Helm of Awakening} ou \card{Mycosynth Lattice} quando seus oponentes são capazes de tirar mais vantagem delas do que você.
Isso também pode ocorrer em cartas que você não espera, como \card{Teferi, Time Raveler}, que diante de uma situação em que você não possui muitas peças de interação, pode permitir que um outro jogador realize um combo sem que os outros dois possam te ajudar a para-lo.


\subsection{Cartas lentas demais:}
Existem muitas cartas no MTG com efeitos poderosos, mas lentas demais para poderem se pagar, o que gera muito espaço para que os oponentes respondam e na maiorias das vezes torna-se apenas um desperdicio de valor ou tempo.
Um exemplo disso é o \card{Xanathar, Guild Kingpin}, uma carta com um efeito visualmente poderoso, capaz de gerar muito valor, porém existem muitos baralhos que ela deve ser cortada, pois embora possua tamanho poder ainda é lenta demais para mostra-lo.
Essas cartas normalmente seguem um padrão, são caras em Mana, não impactam a mesa no turno que entram em jogo e precisam voltar intactas por uma rodada ou mais para fazerem algo, muitas vezes sem serem capazes de se proteger.
Quanto mais turnos se passam, mais o board state tende a ser alterado por rodada, e mais os turnos passam a ser explosivos. Assim cartas que demoram para se pagar ou influenciar esse Board costumam ser cobradas e terem um desenpenho muito abaixo do esperado.
Não ajudando a lidar com os problemas da mesa e colocando alvos em sua cabeça sem que possa se defender corretamente (pois gastou de seu Tempo/Valor na conjuração da mágica), podem ser cartas candidatas a serem removidas do baralho.
Nota-se que essas cartas podem ser muito poderosas quando no Spot correto, mas esse Spot normalmente é difícil de se atingir e deve ser analisado com todas as ideias anteriores em mente. Lembre-se que essas cartas não são avaliáveis em Goldfishing, sendo muito dificeis de ter os problemas percebidos sem o contexto completo do jogo, durante o Goldfishing podem parecer muito melhores do que são.
Pode-se abrir um parênteses também sobre cartas de baixo cmc, mas que precisam de muitos turnos em campo para agir, pois elas também devem ser enquadradas nessa categoria. 


