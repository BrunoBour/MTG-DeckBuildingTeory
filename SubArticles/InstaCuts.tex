Cartas que você deve considerar remover do Deck ex: winmore, fakeCombos, perfectSpots

\subsection{Winmore Cards:}
Possivelmente os Slots mais dificeis de identificar, e por consequencia os mais comuns de se ver em listas iniciais de decks sendo construidos. 
Elas dizem respeito as famosas cartas que só lhe ajudam quando você está muito a frente da mesa, mas não o suficiente para fechar o jogo, o exemplo mais concreto é a carta \card{Unnatural Growth}, vamos trabalhar em cima dela.
Dobrar o poder e resistencia de suas criaturas em todo combate parece muito poderoso, imediatamente vem a cabeça um campo com muitas criaturas imensas e poderosas tendo seu poder dobrado se tornando verdadeiros colossos, mas aí que mora o problema.
Essa é a situação ideal, pois em um campo vazio, ou com uma ou duas criaturas, é apenas uma carta de cmc5 que não lhe traz valor algum, dessa forma é necessário estar muito bem no jogo para que ela possa fazer qualquer coisa.
Nesse caso, a carta ainda não se paga, pois no lugar dela, seria muito melhor um Finisher como \card{Overwhelming Stampede}, capaz de terminar o jogo no mesmo turno em que resolve.
E então uma argumentação pode surgir: "Mas o \card{Unnatural Growth} também pode finalizar o jogo", isso é um fato, ele pode, porém faz a tarefa de forma pífia, uma vez que não concede atropelar e além de ser parada por counterspells como o \card{Overwhelming Stampede}, também pode ser removida por inumeras outras cartas, se tornarndo muiiito mais fácil de parar.
Em resumo, a carta é horrível quando se está atrás, e quando se está na frente \st{parece que está atrás} se torna preferível um bom Finisher, sua contribuição é tornar um campo já com criaturas grandes em criaturas com um corpo ainda maior, mas isso não é a solução para ganhar o jogo.
Esse é o conceito de "Win More", cartas que só são úteis quando se está na frente do jogo e ainda por cima não finalizam ele, tendo assim pouco ou nenhum uso. Elas são cartas fortemente aconselhadas de se remover do baralho.

\subsection{Combos Armadilhas:}
Todo jogador adepto a combos já considerou ou colocou algum combo no deck que não encaixava de forma alguma com o baralho, e isso frequentemente ocorre com a carta \card{Laboratory Maniac} por exemplo, uma carta que não possui absolutamente nenhum uso fora de combos.
Essa opção não necessáriamente é ruim, vemos frequentemente em torneios competitivos o uso de \card{Thassa's Oracle}, que se encaixa nessa mesma descrição, porém existe uma importante distinção entre os contextos, a thoracle é utilizada em baralhos com muitos tutores capazes de realizar o combo com uma consistência absurda.
Tarefa dificil de replicar em baralhos casuais de EDH, o que torna a compra de apenas uma das peças do combo muitas vezes indesejada, já que a carta se torna apenas um draw morto que pode influenciar negativamente suas partidas mais do que você espera.
Por isso sujere-se remover cartas que são uteis única e exclusivante dentro de combos, quando os mesmos não podem ser reproduzidos com consistencia.
Caso seja do seu prazer incluir combos no baralho, opte por utilizar cartas que serão úteis mesmo isoldadas do combo, ou então adicione formas de realizar o combo com consistencia para uma abordagem mais competitiva.


\subsection{Cartas inflexiveis:}
\card{}

\subsection{Cartas que desviam o plano de jogo:}
Com certeza todo jogador evita utilizar cartas que não sinergizam com o deck, ninguém colocaria um \card{Ezuri, Renegade Leader} em um deck de Goblins, mas mais do que isso é necessário observar seu plano de jogo e o Tempo em que seu deck funciona para evitar cartas que desviam dessas ideias.
Utilizarei aqui de exemplo o \card{Simian Spirit Guide}, uma carta muito forte e popular que permite sacrificar valor para acelerar seu Tempo, mas ela perde uma contribuição quando utilizada em um deck que precisa de muito recurso para jogar.
Isso acontece pois um deck que depende de Valor para rodar está trocando esse tão prestigiado recurso por Tempo, uma vantagem que não será explorada com eficiência pelo resto do baralho. Por isso é necessário verificar se suas cartas estão dentro da premissa do deck, analizar se realmente fazem o que o deck precisa que façam.
E caso elas estejam deslocadas do seu plano de jogo, faz muito sentido que remova-as para um ter uma ideia de jogo mais estabelecida.


\subsection{Cartas que auxiliam oponentes:}
\card{Stormfist Crusader} e \card{Helm of Awakening}

\subsection{Cartas lentas demais:}
\card{Xanathar, Guild Kingpin}

