Tempo, é dito por Reid Duke, campeão do Pro Tour Phyrexia, top8 em três outros Pro Tours e vencedor da Magic: The Gathering Online Championship em 2011 (além de alguns outros títulos), ser o recurso mais conectado a vencer ou perder jogos.
Ele o define em sua forma mais básica como a "Board Presence", ou presença de campo, que deriva de como suas permanentes rivalizam com as dos adversários e as consequencias desse embate.
Esse recurso pode ser observado normalmente da forma de pontos de vida ou até em mana e também ser pareado como o outro lado da balança, o "Valor". Reid Duke traz um simples exemplo que ilustra bem esse conceito.
"Algumas vezes é necessário escolher entre ganhar Tempo em troca de Valor ou vice versa, por exemplo, se você gastar um turno para conjurar \card{Divination}, você ganhará Card Advantage(Valor), mas terá gasto um turno sem melhorar sua presença de campo, dessa forma gastará Tempo no processo. Para o segundo caso imagine que sua presença de campo é tão forte que o oponente é forçado a dar Chump Block, perdendo uma criatura importante para preservar vida, assim você recebe valor a frente a partir do Tempo"

Reid Duke traz mais detalhes sobre o conceito de tempo e faz um caminho paralelo entre ele e curva de mana, conceito abordado com maior precisão no seu próprio tópico pois há muito a ser discutido quando se trata de masterizar o uso do Tempo.
Aqui traremos uma abordagem mais lateral, pois quando tratado para formatos um contra um, a ideia de Tempo traz muito consigo um plano agressivo, o que pode ser replicado em commander, mas diz muito mais respeito ao estilo do deck do que o controle do recurso em si.
Dessa forma focaremos na sua outra abordagem, proatividade e 
