Saber o quanto seu baralho depende do comandante é uma parte extremamente importante do Deckbuilding, além de ser um fator que pode ser manipulado a seu favor.
Separaremos esse tópico em diferentes níveis de dependencia (do commander óbvio), e o que deve chamar sua atenção em cada uma delas.
Lembre-se que como dito acima, você pode mudar a categoria em que se encaixa, algumas formas de fazer isso é por exemplo para ir a uma categoria de mais baixa de dependência remover cartas que funcionam apenas com o comandante e adicionar no lugar cartas que interajam com as outras 99.
Ou para aumentar graus de dependência simplesmente fazer o inverso, adicionar cartas que possuem uma extrema sinergia com o comandante



\subsection{Super-Dependência}
Seu deck depende inteiramente do comandante, tendo extrema dificuldade em rodar e ganhar sem ele em campo.
Essa categoria tende a ser a mais delicada de todas elas, tente evita-la caso seu comandante não possua por si só formas de se proteger.
Aqui você precisa destinar inúmeros Slots do baralho apenas a proteger o seu general, utilize peças capazes de dar indestrutibilidade ou resistência a magia em instant speed para desencorajar os oponentes a remove-lo.
Além de cartas que o protejam constantemente, como \card{Guardian Augmenter} e \card{Mother of Runes}, você também pode utilizar cartas que o impeçam de tomar counterspells como \card{Rhythm of the Wild}, e se assegurar de ter mana o suficente para conjura-lo uma segunda ou terceira vez caso ainda assim removido.


\subsection{Alta-Dependência}
Seu deck precisa do comandante para rodar, mas consegue finalizar o jogo mesmo quando ele é removido.
Parecido com a categoria acima você precisará de muitas peças de proteção, a grande diferença é que o período pelo qual precisa ficar na mesa não contempla a partida inteira, assim você pode ceder alguns slots que lidariam com remoções menos pontuais como globais e apenas ser mais cuidadoso na conuração para não "tomar de tabela".
Também é possível utilizar menos ramp ou acesso a mana, já que normalmente precisará conjura-lo pelo menos uma vez a menos que os casos de Super-Dependência.
Ainda assim tenha certeza de mante-lo na mesa pelo tempo necessário e não conjure-o sem ter certeza que sairá impune, pois isso pode lhe custar o jogo.



\subsection{Dependente}
O deck consegue rodar e até finalizar partidas sem nem conjurar o comandante, mas faz isso de forma pífia, se tornarndo muito mais forte com ele em campo.
Esse é o caso em que você pode utilizar muito menos proteções interativas que jogam na pilha, e passar a utilizar apenas as cartas que protegem o general, mas ainda tenham alguma outra função, como por exemplo \card{Swiftfoot Boots} ou \card{Lightning Greaves}.



\subsection{Sub-Dependência}
O deck roda tranquilamente sem o comandante, porém existe uma tarefa que só ele cumpre no deck, como finalizar a partida, proteger seu campo ou dar hate em uma fraqueza do baralho


\subsection{Independente}
O deck funciona sem o comandante sem nenhum problema, existem partidas em que você optará até mesmo por não Casta-lo, ele simplesmente é tratado como uma carta a mais na mão, mas isso não exclui a possibilidade dele ser uma carta poderosa e sinérgica com o baralho.