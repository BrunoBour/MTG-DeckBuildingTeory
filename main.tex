%  Tipo de documento:
\documentclass[12pt, a4paper]{article}

%  Pacotes:
\usepackage[brazil]{babel}
\usepackage{amsmath}
\usepackage{graphicx}


%  Corpo do documento:
\begin{document}
\bgroup\obeylines

\title{Teoria de construção de baralhos}
\author{Bruno/Nathan}
    %  Titulo:
    \maketitle
 
    Esse documento se trata de um compilado de informações sobre "Deckbuilding"
    

    % Sub-Titulos

    \section{Otimização de curva de mana}
    otimizar curva de mana e entende-la para que se possa compreender como jogar com o deck e os turnos de spike

    \section{Otimização de terrenos}
    Como otimizar Lands, controlando cores, números, efeitos e tempo(tapped lands)

    \section{Staples armadilhas}
    Cartas que normalmente são usadas em contextos errados ex: propaganda, lanterna cromática, maskwood nexus

    \section{Cartas que você deve evitar}
    Cartas que você deve considerar remover do Deck ex: winmore, fakeCombos, perfectSpots

    \section{Balanceando os Slots do baralho}
    numero de slots para cada tipo ou funcção de carta dependendo do estilo do deck

    \section{Conjuntos de cartas que pode considerar}
    recomendações de packs de cartas que podem ser includes 


\subsection{\card{Moonsilver Key}:}

\subsection{\card{Crop Rotation}:}

\subsection{\card{Trinket Mage}:}

\subsection{\card{Sun Forger}:}

\subsection{\card{Prime Speaker}:}






    \section{Articulando o plano de jogo}
    traçar um plano de jogo para o deck entender o que ele precisa para jogar e justificar se as cartas contribuem a essa estratégia

    \section{Critérios especificos de cada cor}
    ideias e compreensão dos pontos fortes e fracos das cores e suas combinações, com oque cada cor deve se preocupar e o que deve buscar fazer na maioria dos casos

    \section{Critérios especificos de cada estratégia}
    recomendações para estratégias especificas

    \section{Entendendo e manejando o Tempo do deck}
    Tempo, é dito por Reid Duke, campeão do Pro Tour Phyrexia, top8 em três outros Pro Tours e vencedor da Magic: The Gathering Online Championship em 2011 (além de alguns outros títulos), ser o recurso mais conectado a vencer ou perder jogos.
Ele o define em sua forma mais básica como a "Board Presence", ou presença de campo, que deriva de como suas permanentes rivalizam com as dos adversários e as consequencias desse embate.
Esse recurso pode ser observado normalmente da forma de pontos de vida ou até em mana e também ser pareado como o outro lado da balança, o "Valor". Reid Duke traz um simples exemplo que ilustra bem esse conceito.
"Algumas vezes é necessário escolher entre ganhar Tempo em troca de Valor ou vice versa, por exemplo, se você gastar um turno para conjurar \card{Divination}, você ganhará Card Advantage(Valor), mas terá gasto um turno sem melhorar sua presença de campo, dessa forma gastará Tempo no processo. Para o segundo caso imagine que sua presença de campo é tão forte que o oponente é forçado a dar Chump Block, perdendo uma criatura importante para preservar vida, assim você recebe valor a frente a partir do Tempo"

Reid Duke traz mais detalhes sobre o conceito de tempo e faz um caminho paralelo entre ele e curva de mana, conceito abordado com maior precisão no seu próprio tópico pois há muito a ser discutido quando se trata de masterizar o uso do Tempo.
Aqui traremos uma abordagem mais lateral, pois quando tratado para formatos um contra um, a ideia de Tempo traz muito consigo um plano agressivo, o que pode ser replicado em commander, mas diz muito mais respeito ao estilo do deck do que o controle do recurso em si.
Dessa forma focaremos na sua outra abordagem, proatividade e 


    \section{Entendendo e manejando o Gás do deck}
    entender o "gás" do deck, como garantir mais gás através do valor e o que é necessário sacrificar em prol dele

    \section{Garantindo a robustes do deck}
    robustez do deck, como deixar o deck mais fluido e evitar ficar travado

    \section{Finalização de jogo do baralho}
    como entender o board state necessário para o deck finalizar o game e quais são essas wincons

    \section{Importância de jogar na pilha, interagir e responder}
    como manejar as peças de interações seus tipos, sua importancia e a necessidade de jogar na pilha

    \section{Entendendo e corrigindo as fraquezas do Deck}
    entender a fraqueza do deck e tentar constorna-la ou não deixa-la tão exposta ex: utilizar protects contra globais em creature based decks ou gravehate em mill decks

    \section{Dependencia do comandante}
    registrar a depencia do deck sobre o comandante, o quanto é necessário protege-lo ou conjura-lo 

    \section{Reeavaliando o baralho e seus slots}
    pensamentos sobre as cartas durante o game para refinar o deck, como prestar atenção nas cartas e verificar sua utilidade

    \section{Glossário}
    glossário com os termos ex: card advantage, card select, ramp, etc.

    \end{document}
    
    


