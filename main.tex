%  Tipo de documento:
\documentclass[12pt, a4paper]{book}

%  Pacotes:
\usepackage[brazil]{babel}
\usepackage{amsmath}
\usepackage{graphicx}

%  Corpo do documento:
\begin{document}
\bgroup\obeylines
    %  Titulo:
    \textbf{Teoria de construção de baralhos}
 
    Esse documento se trata de um compilado de informações sobre "Deckbuilding"
    

    % Sub-Titulos
    otimizar curva de mana e entende-la para que se possa compreender como jogar com o deck e os turnos de spike
    Como otimizar Lands, controlando cores, números, efeitos e tempo(tapped lands)
    Cartas que normalmente são usadas em contextos errados



\subsection{\card{Chromatic Lantern}:} A lanterna cromática é a mais comum das pedras de mana de cmc3 utilizadas, essa categoria de pedras de mana tende a ser muito mais fraca do que as genéricas de cmc2, e o motivo disso é na verdade bem óbvio, possuem um custo de mana 50\% maior em relação a concorrência, além de ocuparem o cast do terceiro turno, muitas vezes o turno mais importante do Early Game devido a cartas capazes de pressionar o board com consistencia começarem a surgir, dessa forma, toda mana rock cmc3 precisa de uma ótima compensação para ter seu slot justificado.
Assim, ela parece chamar muita atenção devido a capacidade de corrigir cores, porém em muitos casos isso se trata de uma armadilha, uma vez que pedras de cmc2 normalmente já ajudam nessa correção, gerando duas cores distintas ou até mesmo mana de qualquer cor sem possuir o drawback do custo, isto somado a uma base de mana bem construída deveria já ser o suficiente para atingir os pips necessários. 
Se você ainda acredita que precisa da lanterna para corrigir suas cores provavelmente significa que algo está errado, pois se ela é assim necessária, quando você não compra-la irá fazer o que?! não ter mana para castar as spells?! se esse é o caso, acredito que o seu deck possui entre 4-5 cores e tem grande dependêcia dos pips, situação que a torna justificável, mas ainda não o suficiente, tendo em mente que ela não irá corrigir seus pips em toda partida. 
Em conclusão a lanterna precisa estar acompanhada de outras cartas capazes de corrigir sua mana base no único caso em que é viável, tornando-a uma peça importante na situação supracitada, mas uma má eficiência de mana em qualquer outro contexto.

\subsection{\card{Propaganda}:} Propaganda assim como \card{Ghostly Prision}, possui dois usos principais, evitar swings no early game onde o custo da taxa é muito alto e o proteger de swarm decks.
O segundo uso pode lhe salvar em alguns casos, mas não é a ferramenta ideal para o serviço, pois deixa a desejar na confiabilidade devido ao adversário poder simplesmente remove-lo e então realizar o ataque.
Agora, falando do primeiro caso, cumpre muito bem a promessa, stallando creature based decks e poupando uma imensa quantidade de vida, conquista que pode salvar jogos diante de decks que tendem a oprimir seus life points.
Perfeito, mas então por que se trata de uma armadilha?! Bom, justamente pelos jogadores experienciarem essas boas performances da carta cogitam utiliza-la em decks que não precisam dessa proteção.
O exemplo mais corriqueiro são mid-ranges/aggros que jogam na curva, eles já possuem criaturas poderosas capazes de bloquear, fazendo com que se receba muito menos ataques ou até nenhum.
Nesse contexo a propaganda é uma carta que não possui uso algum, e pode ser muito custosa para o early game, seja pelo fato de ocupar o terceiro turno inteiro ou apenas por ocupar um espaço na mão que poderia ser de uma criatura opressora.
De tal maneira apenas dedique um slot a essa carta se o deck realmente precisa de proteção no early game e tem dificuldades para se manter saudável até o seu spike.

\subsection{\card{Maskwood Nexus}:} Se trata de uma carta muito popular que tende a ser utilizada em um contexto equivocado, pode parecer esquisito a primeira mão, mas o seu espaço não é em tribais com bastante suporte.
O motivo disso é que tribais com bastante suporte normalmente já possuem um número consistente de criaturas fortes dentro da tribo, e por exemplo, não faz sentido algum transformar suas criaturas em elfos se 80\%+ delas já são, ainda mais por um alto custo de mana.
Seu spot para um bom desempenho é muito esepecífico, como por exemplo decks que se importam com mais de uma tribo, situação em que ela pode gerar muito valor. Nos demais casos pode performar muito abaixo do esperado.
    Cartas que você deve considerar remover do Deck ex: winmore, fakeCombos, perfectSpots

\subsection{Winmore Cards:}
Possivelmente os Slots mais dificeis de identificar, e por consequencia os mais comuns de se ver em listas iniciais de decks sendo construidos. 
Elas dizem respeito as famosas cartas que só lhe ajudam quando você está muito a frente da mesa, mas não o suficiente para fechar o jogo, o exemplo mais concreto é a carta \card{Unnatural Growth}, vamos trabalhar em cima dela.
Dobrar o poder e resistencia de suas criaturas em todo combate parece muito poderoso, imediatamente vem a cabeça um campo com muitas criaturas imensas e poderosas tendo seu poder dobrado se tornando verdadeiros colossos, mas aí que mora o problema.
Essa é a situação ideal, pois em um campo vazio, ou com uma ou duas criaturas, é apenas uma carta de cmc5 que não lhe traz valor algum, dessa forma é necessário estar muito bem no jogo para que ela possa fazer qualquer coisa.
Nesse caso, a carta ainda não se paga, pois no lugar dela, seria muito melhor um Finisher como \card{Overwhelming Stampede}, capaz de terminar o jogo no mesmo turno em que resolve.
E então uma argumentação pode surgir: "Mas o \card{Unnatural Growth} também pode finalizar o jogo", isso é um fato, ele pode, porém faz a tarefa de forma pífia, uma vez que não concede atropelar e além de ser parada por counterspells como o \card{Overwhelming Stampede}, também pode ser removida por inumeras outras cartas, se tornarndo muiiito mais fácil de parar.
Em resumo, a carta é horrível quando se está atrás, e quando se está na frente \st{parece que está atrás} se torna preferível um bom Finisher, sua contribuição é tornar um campo já com criaturas grandes em criaturas com um corpo ainda maior, mas isso não é a solução para ganhar o jogo.
Esse é o conceito de "Win More", cartas que só são úteis quando se está na frente do jogo e ainda por cima não finalizam ele, tendo assim pouco ou nenhum uso. Elas são cartas fortemente aconselhadas de se remover do baralho.

\subsection{Combos Armadilhas:}
Todo jogador adepto a combos já considerou ou colocou algum combo no deck que não encaixava de forma alguma com o baralho, e isso frequentemente ocorre com a carta \card{Laboratory Maniac} por exemplo, uma carta que não possui absolutamente nenhum uso fora de combos.
Essa opção não necessáriamente é ruim, vemos frequentemente em torneios competitivos o uso de \card{Thassa's Oracle}, que se encaixa nessa mesma descrição, porém existe uma importante distinção entre os contextos, a thoracle é utilizada em baralhos com muitos tutores capazes de realizar o combo com uma consistência absurda.
Tarefa dificil de replicar em baralhos casuais de EDH, o que torna a compra de apenas uma das peças do combo muitas vezes indesejada, já que a carta se torna apenas um draw morto que pode influenciar negativamente suas partidas mais do que você espera.
Por isso sujere-se remover cartas que são uteis única e exclusivante dentro de combos, quando os mesmos não podem ser reproduzidos com consistencia.
Caso seja do seu prazer incluir combos no baralho, opte por utilizar cartas que serão úteis mesmo isoldadas do combo, ou então adicione formas de realizar o combo com consistencia para uma abordagem mais competitiva.


\subsection{Cartas inflexiveis:}
Cartas que precisam Spots muito específicos para funcionar, se tornando Draws mortos com muita frequência. (Elaborar mais...)

\subsection{Cartas que desviam o plano de jogo:}
Com certeza todo jogador evita utilizar cartas que não sinergizam com o deck, ninguém colocaria um \card{Ezuri, Renegade Leader} em um deck de Goblins, mas mais do que isso é necessário observar seu plano de jogo e o Tempo em que seu deck funciona para evitar cartas que desviam dessas ideias.
Utilizarei aqui de exemplo o \card{Simian Spirit Guide}, uma carta muito forte e popular que permite sacrificar valor para acelerar seu Tempo, mas ela perde uma contribuição quando utilizada em um deck que precisa de muito recurso para jogar.
Isso acontece pois um deck que depende de Valor para rodar está trocando esse tão prestigiado recurso por Tempo, uma vantagem que não será explorada com eficiência pelo resto do baralho. Por isso é necessário verificar se suas cartas estão dentro da premissa do deck, analizar se realmente fazem o que o deck precisa que façam.
E caso elas estejam deslocadas do seu plano de jogo, faz muito sentido que remova-as para um ter uma ideia de jogo mais estabelecida.


\subsection{Cartas que auxiliam oponentes:}
Esses Slots são muitas vezes fáceis de identificar, mas podem ser muito dificeis de ter o impacto mensurado, isso porque costumam ser cartas muito úteis para seu baralho e lhe ajudar bastante.
O que acontece é que podem estar ajudando seus adversários muito mais do que você pensa. Principalmente quando se está iniciando em um TCG pode não se ter a dimensão do quanto dar um Draw para um oponente muda o jogo.
Existem muitas cartas com efeitos de Howling Mine, como a própria que origina esse nome \card{Howling Mine}, e essas peças podem ser o motivo de sua derrota. Embora pareça promissor ganhar mais recurso e ativar efeitos como o de \card{Xyris, the Writhing Storm} ou \card{Nekusar, the Mindrazer}, tem muitas coisas que é preciso considerar.
Primeiramente é importante perceber que você está gastanto do seu recurso e tempo para colocar essas cartas na mesa, enquanto os oponentes se beneficiam do efeito sem nenhum Drawback, e por segundo, mas ainda mais importante, considerando que você está contra três outros jogadores, note que enquanto você compra uma carta, seus oponentes compram juntos três delas.
Dessa forma se torna muito difícil vencer spots de Archenemy, dando muita margem para que jogadores mais fracos na mesa possam tirar uma vitória "do nada". Permitir essa imprevisibilidade para seus oponentes é um grande erro quando o seu objetivo é ganhar.
Os efeitos de Howling Mine são os mais fáceis de verificar esse problema, mas ainda é comum de acontecer em cartas como \card{Helm of Awakening} ou \card{Mycosynth Lattice} quando seus oponentes são capazes de tirar mais vantagem delas do que você.
Isso também pode ocorrer em cartas que você não espera, como \card{Teferi, Time Raveler}, que diante de uma situação em que você não possui muitas peças de interação, pode permitir que um outro jogador realize um combo sem que os outros dois possam te ajudar a para-lo.


\subsection{Cartas lentas demais:}
Existem muitas cartas no MTG com efeitos poderosos, mas lentas demais para poderem se pagar, o que gera muito espaço para que os oponentes respondam e na maiorias das vezes torna-se apenas um desperdicio de valor ou tempo.
Um exemplo disso é o \card{Xanathar, Guild Kingpin}, uma carta com um efeito visualmente poderoso, capaz de gerar muito valor, porém existem muitos baralhos que ela deve ser cortada, pois embora possua tamanho poder ainda é lenta demais para mostra-lo.
Essas cartas normalmente seguem um padrão, são caras em Mana, não impactam a mesa no turno que entram em jogo e precisam voltar intactas por uma rodada ou mais para fazerem algo, muitas vezes sem serem capazes de se proteger.
Quanto mais turnos se passam, mais o board state tende a ser alterado por rodada, e mais os turnos passam a ser explosivos. Assim cartas que demoram para se pagar ou influenciar esse Board costumam ser cobradas e terem um desenpenho muito abaixo do esperado.
Não ajudando a lidar com os problemas da mesa e colocando alvos em sua cabeça sem que possa se defender corretamente (pois gastou de seu Tempo/Valor na conjuração da mágica), podem ser cartas candidatas a serem removidas do baralho.
Nota-se que essas cartas podem ser muito poderosas quando no Spot correto, mas esse Spot normalmente é difícil de se atingir e deve ser analisado com todas as ideias anteriores em mente. Lembre-se que essas cartas não são avaliáveis em Goldfishing, sendo muito dificeis de ter os problemas percebidos sem o contexto completo do jogo, durante o Goldfishing podem parecer muito melhores do que são.
Pode-se abrir um parênteses também sobre cartas de baixo cmc, mas que precisam de muitos turnos em campo para agir, pois elas também devem ser enquadradas nessa categoria. 



    numero de slots para cada tipo ou funcção de carta dependendo do estilo do deck
    recomendações de packs de cartas que podem ser includes 


\subsection{\card{Moonsilver Key}:}

\subsection{\card{Crop Rotation}:}

\subsection{\card{Trinket Mage}:}

\subsection{\card{Sun Forger}:}

\subsection{\card{Prime Speaker}:}





    traçar um plano de jogo para o deck entender o que ele precisa para jogar e justificar se as cartas contribuem a essa estratégia
    ideias e compreensão dos pontos fortes e fracos das cores e suas combinações, com oque cada cor deve se preocupar e o que deve buscar fazer na maioria dos casos
    \input{SubArticles/SpecificStrategies}

    Tempo, é dito por Reid Duke, campeão do Pro Tour Phyrexia, top8 em três outros Pro Tours e vencedor da Magic: The Gathering Online Championship em 2011 (além de alguns outros títulos), ser o recurso mais conectado a vencer ou perder jogos.
Ele o define em sua forma mais básica como a "Board Presence", ou presença de campo, que deriva de como suas permanentes rivalizam com as dos adversários e as consequencias desse embate.
Esse recurso pode ser observado normalmente da forma de pontos de vida ou até em mana e também ser pareado como o outro lado da balança, o "Valor". Reid Duke traz um simples exemplo que ilustra bem esse conceito.
"Algumas vezes é necessário escolher entre ganhar Tempo em troca de Valor ou vice versa, por exemplo, se você gastar um turno para conjurar \card{Divination}, você ganhará Card Advantage(Valor), mas terá gasto um turno sem melhorar sua presença de campo, dessa forma gastará Tempo no processo. Para o segundo caso imagine que sua presença de campo é tão forte que o oponente é forçado a dar Chump Block, perdendo uma criatura importante para preservar vida, assim você recebe valor a frente a partir do Tempo"

Reid Duke traz mais detalhes sobre o conceito de tempo e faz um caminho paralelo entre ele e curva de mana, conceito abordado com maior precisão no seu próprio tópico pois há muito a ser discutido quando se trata de masterizar o uso do Tempo.
Aqui traremos uma abordagem mais lateral, pois quando tratado para formatos um contra um, a ideia de Tempo traz muito consigo um plano agressivo, o que pode ser replicado em commander, mas diz muito mais respeito ao estilo do deck do que o controle do recurso em si.
Dessa forma focaremos na sua outra abordagem, proatividade e 

    entender o "gás" do deck, como garantir mais gás através do valor e o que é necessário sacrificar em prol dele
    robustez do deck, como deixar o deck mais fluido e evitar ficar travado
    como entender o board state necessário para o deck finalizar o game e quais são essas wincons
    como manejar as peças de interações seus tipos, sua importancia e a necessidade de jogar na pilha
    entender a fraqueza do deck e tentar constorna-la ou não deixa-la tão exposta ex: utilizar protects contra globais em creature based decks ou gravehate em mill decks
    Saber o quanto seu baralho depende do comandante é uma parte extremamente importante do Deckbuilding, além de ser um fator que pode ser manipulado a seu favor.
Separaremos esse tópico em diferentes níveis de dependencia (do commander óbvio), e o que deve chamar sua atenção em cada uma delas.
Lembre-se que como dito acima, você pode mudar a categoria em que se encaixa, algumas formas de fazer isso é por exemplo para ir a uma categoria de mais baixa de dependência remover cartas que funcionam apenas com o comandante e adicionar no lugar cartas que interajam com as outras 99.
Ou para aumentar graus de dependência simplesmente fazer o inverso, adicionar cartas que possuem uma extrema sinergia com o comandante



\subsection{Super-Dependência}
Seu deck depende inteiramente do comandante, tendo extrema dificuldade em rodar e ganhar sem ele em campo.
Essa categoria tende a ser a mais delicada de todas elas, tente evita-la caso seu comandante não possua por si só formas de se proteger.
Aqui você precisa destinar inúmeros Slots do baralho apenas a proteger o seu general, utilize peças capazes de dar indestrutibilidade ou resistência a magia em instant speed para desencorajar os oponentes a remove-lo.
Além de cartas que o protejam constantemente, como \card{Guardian Augmenter} e \card{Mother of Runes}, você também pode utilizar cartas que o impeçam de tomar counterspells como \card{Rhythm of the Wild}, e se assegurar de ter mana o suficente para conjura-lo uma segunda ou terceira vez caso ainda assim removido.


\subsection{Alta-Dependência}
Seu deck precisa do comandante para rodar, mas consegue finalizar o jogo mesmo quando ele é removido.
Parecido com a categoria acima você precisará de muitas peças de proteção, a grande diferença é que o período pelo qual precisa ficar na mesa não contempla a partida inteira, assim você pode ceder alguns slots que lidariam com remoções menos pontuais como globais e apenas ser mais cuidadoso na conuração para não "tomar de tabela".
Também é possível utilizar menos ramp ou acesso a mana, já que normalmente precisará conjura-lo pelo menos uma vez a menos que os casos de Super-Dependência.
Ainda assim tenha certeza de mante-lo na mesa pelo tempo necessário e não conjure-o sem ter certeza que sairá impune, pois isso pode lhe custar o jogo.



\subsection{Dependente}
O deck consegue rodar e até finalizar partidas sem nem conjurar o comandante, mas faz isso de forma pífia, se tornarndo muito mais forte com ele em campo.
Esse é o caso em que você pode utilizar muito menos proteções interativas que jogam na pilha, e passar a utilizar apenas as cartas que protegem o general, mas ainda tenham alguma outra função, como por exemplo \card{Swiftfoot Boots} ou \card{Lightning Greaves}.



\subsection{Sub-Dependência}
O deck roda tranquilamente sem o comandante, porém existe uma tarefa que só ele cumpre no deck, como finalizar a partida, proteger seu campo ou dar hate em uma fraqueza do baralho


\subsection{Independente}
O deck funciona sem o comandante sem nenhum problema, existem partidas em que você optará até mesmo por não Casta-lo, ele simplesmente é tratado como uma carta a mais na mão, mas isso não exclui a possibilidade dele ser uma carta poderosa e sinérgica com o baralho.
    pensamentos sobre as cartas durante o game para refinar o deck, como prestar atenção nas cartas e verificar sua utilidade
    glossário com os termos ex: card advantage, card select, ramp, etc.
    \end{document}
    
    


